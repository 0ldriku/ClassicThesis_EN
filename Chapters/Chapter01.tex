\chapter{Introduction}



This template is designed to help my colleagues write their theses efficiently. It is based on the \texttt{classicthesis} package version 4.8 (\url{https://ctan.org/pkg/classicthesis?lang=en}), with several customizations tailored to our writing needs.



This chapter provides an overview of the template's features and basic usage instructions. \autoref{ch:examples} demonstrates various \LaTeX{} features with practical examples, while \autoref{ch:settings} offers detailed explanations of template configuration options.

\section{Quick Start (TL;DR)}

\textit{Just want to start writing?} Here are the essential steps:

\begin{enumerate}
    \item \textit{Add your info}: Edit \texttt{classicthesis-config.tex} (line 54+) - fill in your name, title, department, etc.

    \item \textit{Write your content}: Put your chapters in \texttt{Chapters/Chapter01.tex}, \texttt{Chapter02.tex}, etc.

    \item \textit{Build your references}: Add bib entries to \texttt{Bibliography.bib} (or \texttt{part1.bib}, \texttt{part2.bib}, etc.)

    \item \textit{Add images}: Put figures in \texttt{gfx/} folder

    \item \textit{Compile}: Open the Menu (top left) and set the Compiler to pdfLaTeX.

\end{enumerate}

That's it! For detailed explanations, continue reading below.


\vspace{1cm}
\textbf{If your thesis are mainly in Japanese, use below.}

\begin{center}
    \url{https://github.com/0ldriku/ClassicThesis_JA}
\end{center}


\section{Japanese Support}


If your thesis is primarily in English with minimal Japanese content, you may use pdfLaTeX by adding the following to your preamble:

\begin{lstlisting}
%*************************************
\usepackage{CJKutf8}
%*************************************
\end{lstlisting}

Then, in your document body, wrap Japanese text as follows:

\begin{lstlisting}
%*************************************
\begin{CJK}{UTF8}{ipxm}
    Put your Japanese text here.
\end{CJK}
%*************************************
\end{lstlisting}

For more information about Japanese support in \LaTeX, see \url{https://www.overleaf.com/learn/latex/Japanese}.

\section{Citation Style}


Build your bibliography entries following the standard BibTeX format. Below are examples of common entry types: 

\begin{lstlisting}

@incollection{fiorella2022,
  author    = {Fiorella, Logan and Mayer, Richard E.},
  title     = {The Generative Activity Principle in Multimedia Learning},
  booktitle = {The Cambridge Handbook of Multimedia Learning},
  editor    = {Mayer, Richard E. and Fiorella, Logan},
  edition   = {3},
  publisher = {Cambridge University Press},
  address   = {Cambridge},
  year      = {2022},
  pages     = {339--350},
  doi       = {10.1017/9781108894333.036}
}

@article{lusato2025,
  author    = {Lu, Jialiang and Sato, Reiko},
  title     = {Linguistic dimensions of comprehensibility and perceived fluency in {L2} speech across tasks of varying complexity},
  journal   = {Journal of Second Language Pronunciation},
  volume    = {11},
  number    = {2},
  year      = {2025},
  pages     = {240--266},
  doi       = {10.1075/jslp.24057.lu}
}

@book{mayer2021,
  author    = {Mayer, Richard E.},
  title     = {Multimedia Learning},
  edition   = {3},
  publisher = {Cambridge University Press},
  address   = {Cambridge},
  year      = {2020},
  doi       = {10.1017/9781316941355}
}

@INPROCEEDINGS{Miede2011,
    author = {Andr{\'e} Miede and G\"{o}khan \c{S}im\c{s}ek and Stefan Schulte
    and Abawi, Daniel F. and Julian Eckert and Ralf Steinmetz},
    title = {{R}evealing {B}usiness {R}elationships -- {E}avesdropping {C}ross-organizational
    {C}ollaboration in the {I}nternet of {S}ervices},
    booktitle = {Proceedings of the Tenth International Conference Wirtschaftsinformatik
    (WI 2011)},
    year = {2011},
    volume = {2},
    pages = {1083--1092},
    isbn = {978-1-4467-9236-0}
}
\end{lstlisting}

