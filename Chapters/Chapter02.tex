\chapter{Usage Examples}\label{ch:examples}

This chapter serves as a guide and template for using this thesis style. It demonstrates the most common elements you will need: sectioning, citations, figures, tables, and math.

\section{Text and Structure}

\subsection{Headers}

\begin{lstlisting}
%*************************************
\part{}
\chapter{} 
\section{}
\subsection{}
\subsubsection{}
\paragraph{}
\subparagraph{} 
%*************************************



\end{lstlisting}

\subsection{Font Size}
LaTeX provides a hierarchy of font size commands to accommodate various typographical needs within a document. The ten standard font size commands, detailed in \autoref{tab:fontsizes}, range from \texttt{\textbackslash tiny} for footnotes and subscripts to \texttt{\textbackslash Huge} for major headings and titles. These commands are relative to the document's base font size (specified in the document class options) and ensure consistent scaling throughout the document.





\begin{table}[ht]
\centering

\caption{LaTeX Font Size Commands Reference}
\renewcommand{\arraystretch}{1.8}
% Modern column definition with better spacing
\begin{tabular}{@{\extracolsep{8pt}}l l@{}}
\toprule
\rowcolor{headerblue}
\color{white}\textbf{Command} & 
\color{white}\textbf{Example Output} \\
\midrule
\rowcolor{lightgray}
\texttt{\textbackslash tiny} & {\tiny This is tiny text} \\
\texttt{\textbackslash scriptsize} & {\scriptsize This is scriptsize text} \\
\rowcolor{lightgray}
\texttt{\textbackslash footnotesize} & {\footnotesize This is footnotesize text} \\
\texttt{\textbackslash small} & {\small This is small text} \\
\rowcolor{lightgray}
\texttt{\textbackslash normalsize} & {\normalsize This is normalsize text} \\
\texttt{\textbackslash large} & {\large This is large text} \\
\rowcolor{lightgray}
\texttt{\textbackslash Large} & {\Large This is Large text} \\
\texttt{\textbackslash LARGE} & {\LARGE This is LARGE text} \\
\rowcolor{lightgray}
\texttt{\textbackslash huge} & {\huge This is huge text} \\
\texttt{\textbackslash Huge} & {\Huge This is Huge text} \\
\bottomrule
\label{tab:fontsizes}
\end{tabular}

\end{table}


Here is a single paragraph that demonstrates the relative scales of standard LaTeX font commands. We start with {\tiny tiny text for fine details}, move up to {\scriptsize script size usually for subscripts}, and then {\footnotesize footnote size}. 

Gradually, we reach {\small small text}, before returning to {\normalsize the default normal size}. To emphasize points, we can use {\large large text}, {\Large larger text for sub-titles}, or {\LARGE even larger text}. Finally, for major impacts, we use {\huge huge} and {\Huge massive sizes}.



\subsection{Formatting}

Examples: \textit{Italics}, \textbf{bold}, \spacedallcaps{All Caps}, \textsc{Small
Caps}, \spacedlowsmallcaps{Low Small Caps}.

Acronym testing: \ac{UML} -- \acs{UML} -- \acf{UML} -- \acp{UML}



\subsection{Lists}
Here is an itemized list:
\begin{itemize}
    \item First item
    \item Second item
\end{itemize}

Here is an enumerated list:
\begin{enumerate}
    \item First step
    \item Second step
\end{enumerate}

Here is an description list:

\begin{description}
    \item[item 1:] First step
    \item[item 2:] Second step
\end{description}


\subsection{Quote}
\begin{quote}
    Similar patterns regarding information retention have been observed in cross-platform rendering engines.
\end{quote}

\section{Citations}
This template uses \texttt{biblatex} with APA style.

\begin{description}
    
    \item[Parenthetical citation:] \parencite{fiorella2022}.
    \item[Textual citation:] \textcite{fiorella2022} found that...
    \item[Multiple citations:] \parencite{fiorella2022}.
    \item[Cite year, author:] \citeauthor{fiorella2022}'s (\citeyear{fiorella2022})
\end{description}



The cafeteria was loud, but my focus was singular. I stared at the legendary dish before me: the Tokyo Tech Power Bowl. While many consume it blindly, I sought to understand its structural integrity.

As \textcite{Miede2011} famously argued in their seminal paper on cafeteria dynamics, the precise allocation of Mizuna greens to grilled pork is not merely a culinary choice, but a mathematical necessity. I picked up my chopsticks, ready to verify their findings.

To analyze the bowl properly, I needed to minimize cognitive load. I applied the principles of \citeauthor{mayer2021}, who established in \citeyear{mayer2021} that people learn better from words and pictures than from words alone. Therefore, I took a picture of the bowl before eating it.

Digging deeper into the rice, I suspected a hidden network of flavors. It felt almost illicit, like the "eavesdropping" techniques described by \textcite{Miede2011} in the context of business relationships. Was the garlic sauce communicating secretly with the pork?

I also considered the possibility of a secret ingredient. \textcite{Miede2011} recently published a comprehensive guide on bananas. Could there be a banana hidden in the Power Bowl? A quick taste test confirmed: definitely not.


The energy density of this meal is high. According to the generative activity principle discussed by \textcite{fiorella2022}, learning—or in this case, digestion—is a generative process.

However, one must be careful not to confuse correlation with causation. As \textcite{Miede2011} warn in their study on dummy variables, what looks like a piece of pork might actually be a cleverly disguised piece of fried garlic (a culinary "dummy" variable, if you will).


The mystery remains partially unsolved. While \citeauthor{Miede2011}'s theory on the Mizuna-Pork ratio holds true (\citeyear{Miede2011}), the emotional impact of the Power Bowl transcends academic citation.

\section{Symbols}

Refer to \autoref{tab:symbols} for the correct input codes, noting that the most common cause of compilation errors is the unescaped use of the ampersand (\&).


\begin{table}[ht]
\centering
\caption{Common LaTeX Symbols Reference}
\renewcommand{\arraystretch}{1.5}
% Modern column definition with better spacing
\begin{tabular}{@{\extracolsep{4pt}}l c l c l c@{}}
\toprule
\rowcolor{headerblue}
\color{white}\textbf{Input} & \color{white}\textbf{Output} & 
\color{white}\textbf{Input} & \color{white}\textbf{Output} & 
\color{white}\textbf{Input} & \color{white}\textbf{Output} \\
\midrule
\rowcolor{lightgray}
\texttt{\textbackslash\%} & \% & 
\texttt{\textbackslash\$} & \$ & 
\texttt{\textbackslash\&} & \& \\
\texttt{\textbackslash\{} & \{ & 
\texttt{\textbackslash\}} & \} & 
\texttt{\textbackslash\#} & \# \\
\midrule
\rowcolor{lightgray}
\texttt{\$\textbackslash alpha\$} & $\alpha$ & 
\texttt{\$\textbackslash theta\$} & $\theta$ & 
\texttt{\$\textbackslash pi\$} & $\pi$ \\

\texttt{\$\textbackslash Gamma\$} & $\Gamma$ & 
\texttt{\$\textbackslash Delta\$} & $\Delta$ & 
\texttt{\$\textbackslash Phi\$} & $\Phi$ \\
\bottomrule
\label{tab:symbols}
\end{tabular}
\end{table}

\section{Page Break}

Commands to perform a page break include {\textbackslash}pagebreak, {\textbackslash}newpage, and {\textbackslash}clearpage. Although the specific usage differs for each command, they are basically used to initiate a new page.

\newpage

\section{Figures}
Figures should be placed in the \texttt{gfx/} folder.

\begin{figure}[htbp]
    \centering
    % Using a standard example image if your specific image isn't available
    \includegraphics[width=0.26\textwidth]{gfx/example_1.jpg} 
    \caption{Example of a single figure.}
    \label{fig:example-single}
\end{figure}


\begin{figure}[htbp]
    \centering
    \begin{minipage}[b]{0.45\textwidth}
        \centering
        \includegraphics[width=\textwidth]{gfx/example_1.jpg}
        \caption*{(a) First figure}
    \end{minipage}
    \hfill
    \begin{minipage}[b]{0.45\textwidth}
        \centering
        \includegraphics[width=\textwidth]{gfx/example_2.jpg}
        \caption*{(b) Second figure}
    \end{minipage}
    
    \vspace{1em}
    
    \begin{minipage}[b]{0.45\textwidth}
        \centering
        \includegraphics[width=\textwidth]{gfx/example_3.jpg}
        \caption*{(c) Third figure}
    \end{minipage}
    \hfill
    \begin{minipage}[b]{0.45\textwidth}
        \centering
        \includegraphics[width=\textwidth]{gfx/example_4.jpg}
        \caption*{(d) Fourth figure}
    \end{minipage}
    
    \caption{Overall caption for all four figures}
    \label{fig:all}
\end{figure}


\newpage

\section{Tables}
Use \texttt{booktabs} for professional quality tables and \texttt{tabularx} for width control.

\begin{table}[ht]
    \centering
    \footnotesize
    \caption{Example table with booktabs}
    \label{tab:example}
    \begin{tabularx}{\textwidth}{lX}
        \toprule
        \textbf{Column 1} & \textbf{Column 2 (Flexible Width)} \\
        \midrule
        Item A & Description of Item A which might be long and wrap to the next line. \\
        Item B & Description of Item B. \\
        \bottomrule
    \end{tabularx}
\end{table}


\begin{table}[ht]
\centering
\footnotesize
\begin{tabular}{lccr}
\toprule

\textbf{Product} & \textbf{Category} & \textbf{Stock} & \textbf{Price (\$)} \\
\midrule
MacBook Pro 16" & Laptop & 45 & 2,499 \\
Dell XPS 15 & Laptop & 32 & 1,799 \\
ThinkPad X1 Carbon & Laptop & 28 & 1,899 \\
\midrule
Magic Mouse 2 & Accessory & 156 & 79 \\
Logitech MX Master 3 & Accessory & 203 & 99 \\
Razer DeathAdder V2 & Accessory & 87 & 69 \\
\midrule
Mechanical Keyboard & Peripheral & 64 & 129 \\
Wireless Keyboard & Peripheral & 91 & 59 \\
Gaming Keyboard RGB & Peripheral & 43 & 159 \\
\midrule
27" 4K Monitor & Display & 38 & 549 \\
34" Ultrawide Monitor & Display & 22 & 899 \\
32" Gaming Monitor & Display & 29 & 699 \\
\midrule
USB-C Hub & Adapter & 245 & 49 \\
Thunderbolt Dock & Adapter & 67 & 279 \\
HDMI Cable 10ft & Cable & 412 & 15 \\
\bottomrule
\end{tabular}
\caption{Electronics inventory with booktabs styling}
\label{tab:inventory}
\end{table}


\begin{table*}[ht]
  \centering
  \setlength{\tabcolsep}{1.5pt} % default is 6pt

  \caption{An example of a long table}
  \label{tab:metrics}
  \footnotesize
  \setlength{\tabcolsep}{2pt}
  \begin{threeparttable}
    % Use S columns defined with Mean(SD) format
    % We need to find the max format for each column across the *whole* table
    % Col 1 (Hist 1 Sim, Sci 1 Sim): Max is 2.2(2.2)
    % Col 2 (Hist 1 Com, Sci 1 Com): Max is 3.2(2.2) (due to '100')
    % Col 3 (Hist 2 Sim, Sci 2 Sim): Max is 2.2(2.2) (due to 15.81)
    % Col 4 (Hist 2 Com, Sci 2 Com): Max is 2.2(2.2)
    \begin{tabular*}{\textwidth}{@{\extracolsep{\fill}}
      l@{\extracolsep{\fill}} 
      S[table-format=2.2(2.2)]
      S[table-format=3.2(2.2)]
      S[table-format=2.2(2.2)]
      S[table-format=2.2(2.2)]
    }
      \toprule
      % History 1 and 2
      & \multicolumn{2}{c}{\textsc{History 1}}
      & \multicolumn{2}{c}{\textsc{History 2}} \\
      \cmidrule(lr){2-3}\cmidrule(lr){4-5}
      % These are already \multicolumn, so they override 'S' and center
      & \multicolumn{1}{c}{\textsc{Simple}} & \multicolumn{1}{c}{\textsc{Complex}}
      & \multicolumn{1}{c}{\textsc{Simple}} & \multicolumn{1}{c}{\textsc{Complex}} \\
    \midrule
    %------ Syntactic block ------%
    \addlinespace
    \multicolumn{5}{@{}l}{\textit{Syntactic Complexity}} \\
    Sentences (\emph{N})                  & 27   & 28   & 30   & 29   \\
    Sentence length (tokens)              & 24.44 (12.52) & 46.07 (22.36) & 24.77 (9.06) & 47.83 (23.11) \\
    Max sentence length                   & 54   & 100  & 43   & 95   \\
    Parse-tree height                     & 4.15 (2.03) & 7.04 (3.01) & 4.43 (1.57) & 7.34 (3.14) \\
    Mean dependency distance              & 2.62 (0.59) & 3.27 (0.88) & 2.82 (0.40) & 3.19 (0.50) \\

    %------ Lexical block ------%
    \addlinespace
    \multicolumn{5}{@{}l}{\textit{Lexical Complexity}} \\
    High-frequency tokens (\%) & 17.74 & 14.45 & 17.01 & 13.90 \\
    Mid-frequency tokens (\%)  & 22.91 & 26.38 & 19.50 & 21.07 \\
    Low-frequency tokens (\%)  &  3.92 &  5.77 &  5.94 &  7.75 \\
    \midrule
    %------ Science 1 and 2 ------%
    & \multicolumn{2}{c}{\textsc{Science 1}}
      & \multicolumn{2}{c}{\textsc{Science 2}} \\
      \cmidrule(lr){2-3}\cmidrule(lr){4-5}
      & \multicolumn{1}{c}{\textsc{Simple}} & \multicolumn{1}{c}{\textsc{Complex}}
      & \multicolumn{1}{c}{\textsc{Simple}} & \multicolumn{1}{c}{\textsc{Complex}} \\
    \midrule
    %------ Syntactic block ------%
    \addlinespace
    \multicolumn{5}{@{}l}{\textit{Syntactic Complexity}} \\
    Sentences (\emph{N})                  & 26   & 22   & 23   & 25   \\
    Sentence length (tokens)              & 26.31 (7.91) & 57.23 (16.63) & 32.35 (15.81) & 50.32 (19.15) \\
    Max sentence length                   & 45   & 94   & 87   & 99   \\
    Parse-tree height                     & 5.23 (1.34) & 8.82 (2.30) & 5.61 (1.83) & 8.76 (2.85) \\
    Mean dependency distance              & 2.90 (0.62) & 2.99 (0.76) & 2.91 (0.51) & 3.12 (0.66) \\

    %------ Lexical block ------%
    \addlinespace
    \multicolumn{5}{@{}l}{\textit{Lexical Complexity}} \\
    High-frequency tokens (\%) & 14.24 & 13.40 & 14.37 & 16.80 \\
    Mid-frequency tokens (\%)  & 26.36 & 27.05 & 27.75 & 29.30 \\
    Low-frequency tokens (\%)  &  8.94 &  8.77 &  4.46 &  4.46 \\
    \bottomrule
    \end{tabular*}
    \begin{tablenotes}[flushleft]
      \footnotesize
      \item \emph{Note.} Values in parentheses represent standard deviations. Most metrics show means across sentences, except counts and maximum values.
    \end{tablenotes}
  \end{threeparttable}
\end{table*}

\clearpage

\section{Mathematics}
Equations can be inline $E=mc^2$ or displayed:

\begin{equation}
    \label{eq:example}
    f(x) = \int_{-\infty}^{\infty} \hat{f}(\xi)\,e^{2\pi i \xi x} \,d\xi
\end{equation}

For multi-line equations, use \texttt{align}:
\begin{align}
    a &= b + c \\
    &= d + e
\end{align}

\section{Cross-Referencing}
You can reference chapters (\autoref{ch:examples}), figures (\autoref{fig:example-single}), tables (\autoref{tab:example}), and equations (\autoref{eq:example}) automatically.


\section{Japanese}

\begin{CJK}{UTF8}{ipxm}
吾輩は猫である。
\end{CJK}